\documentclass[10pt,courier]{navymemo}

\author{Dennis Evangelista}
\title{Letter of recommendation ICO MIDN 1/C E Hiler}
\navysubj{Letter of recommendation ICO MIDN 1/C E Hiler}
%\navyfiling{5200.1}
%\navyserial{16-0001}
\date{\today}
%\navymarking{UNCLASSIFIED}

\usepackage{designature}

\begin{document}
\makedateblock{}

\MEMORANDUM{}

\begin{navyletterheader}
\navyfrom{Asst Prof Evangelista}
\navyto{Inter-service Comissioning Board}
\navyskip{}%
\navysubjline{}%
\end{navyletterheader}

%For main points I would like you to try and include my ealy initiative on  the cross commissioning program, the length of my service and commitment to the navy, and lastly the accelerated pace that I have taken classes at the academy and in my major. I feel these three items best encompass the image of commitment and initiative I am trying to show. 

%As far as the argument I am putting forward, what I have focussed on in my personal statement, and the view that I have gotten from people who have been involved in the cross commissioning program in the past has been rather than convincing either side it would be better for them to either let me go or take me, that my argument should be that the federal services as a whole would get better use out of what I bring to the table in the Coast Guard rather than the Navy. For example, I want to go into inspections in the Coast Guard, specifically the inspections of foreign flagged ships entering US ports. With my education in Naval Architecture, without further higher education my skills are more readily useful in the Coast Guard than anything the Navy could offer, such as EDO which requires at minimum a Masters.

%The only thing That I would change is downplay or remove references to a parallel Navy job I could do such as EDO. While I am putting in for Sub EDO in my Navy service selection, showing that the navy doesn't have something similar to what I want to do is one of the big hurdles I am working against. Instead maybe play up the past 2 years of personal effort and research I have put into seeing if the CG is a good fit for myself. I qualified ERUL on TS 635 at NPTU Charleston. I did not put in for SPU as by the time the application for that spot came due I had already been accepted to NAPS. Unfortunately my qualification has since lapsed.


\section{} 
I support MIDN 1/C E Hiler's request to cross-commission into the United States Coast Guard. I am an assistant professor in the Department of Weapons, Robotics, and Control Engineering. I have known Mr.~Hiler for a year as a faculty advisor to the Naval Academy Thinkers, Atheists, and Secularists group, of which he is a founding member. In that context, I am impressed by his deeply questioning, challenging attitude; Hiler constantly seeks for what is ethical and moral, and pursues nobility in every thought and action. I am also familiar with Hiler's prior enlisted nuclear service as a Machinist's Mate; I served as a Naval Reactors engineer assigned to Naval Sea Systems Command 08, where I worked on reactor plant fluid systems and new concept submarine and plant designs. 

\section{}
I have not had the pleasure of teaching Mr.~Hiler in one of my classes, but he is doing well as a Naval Architecture and Marine Engineering major and is ahead of his matrix. Students who completed Nuclear Power School and qualified on an operating nuclear reactor during prototype training, like Hiler, invariably do well in the classroom. On every administrative task I have seen him engaged in, including ECA paperwork and items related to this cross-commissioning paperwork, he has taken the initiative. I believe all of these bode well for his future as an officer wherever he serves.

\section{}
I understand Mr.~Hiler wishes to use his Naval Architecture and Marine Engineering major and hands on engineering experience in service of the United States Coast Guard, in the vital role of inspection and survey, especially of foreign flag ships, to ensure force protection, homeland security, and compliance with customs, bio-security, marine safety, and environmental regulations. I understand Hiler sees the Coast Guard mission as the "shield" protecting American waters, vice the maritime "sword" of the Navy, projecting power. The character of one's service needs to fit the man; I try to encourage nuclear accessions at USNA and often do practice nuclear power interviews for midshipmen, but I find many who were prior-enlisted nuclear field petty officers ultimately choose other commissioning paths. Highly trained individuals like Hiler are undoubtedly of most benefit to the federal service in roles that fit their ethos.
%\respectfully{}
\noclosing{}
\signature{\includesignature}
\signature{D EVANGELISTA}

\end{document}



%Unfortunately no, that I have found. My personal motivation for wanting to do the Coast Guard is what I see as the Shield vs. the Sword. To me the Navy is the nation's maritime sword, projecting power and conducting war. The Coast Guard is the maritime Shield, keeping American waters safe with law enforcement, Search and rescue, regulation enforcement, and signal maintenance. I would rather my life be conducted as the shield than the sword.

%I do have a mentor. He was a CG Lieutenant who taught here my youngster year. He got out after then, but he inspired me  to start down this path and has been helping me along for the past 2 years. 

%1. Actually the shield vs sword thing is great - that's sort of the thing I am looking for regarding why you should be a CG Ensign vice a Navy Ensign. Service works best when it fits the underlying personality of the man involved - I would have made a shitty marine, and being a nuke was the right way to have me serve (sending me to a boat would have been better but NR does what it wants)... Knowing that about you makes the swap very sensible. Do you want me to work in something like that? 

%2. Feel free to shoot my drafts by your CG mentor if you think it will help. I have never done a letter for someone seeking to cross-commisssion and we want to maximize your chances. 
