\documentclass[10pt,courier]{navymemo}

\author{Dennis Evangelista}
\title{Letter of recommendation ICO MIDN 1/C E Hiler}
\navysubj{Letter of recommendation ICO MIDN 1/C E Hiler}
%\navyfiling{5200.1}
%\navyserial{16-0001}
\date{\today}
%\navymarking{UNCLASSIFIED}

\usepackage{designature}

\begin{document}
\makedateblock{}

\MEMORANDUM{}

\begin{navyletterheader}
\navyfrom{Asst Prof Evangelista}
\navyto{Whom?}
\navyskip{}%
\navysubjline{}%
\end{navyletterheader}

%For main points I would like you to try and include my ealy initiative on  the cross commissioning program, the length of my service and commitment to the navy, and lastly the accelerated pace that I have taken classes at the academy and in my major. I feel these three items best encompass the image of commitment and initiative I am trying to show. 

%As far as the argument I am putting forward, what I have focussed on in my personal statement, and the view that I have gotten from people who have been involved in the cross commissioning program in the past has been rather than convincing either side it would be better for them to either let me go or take me, that my argument should be that the federal services as a whole would get better use out of what I bring to the table in the Coast Guard rather than the Navy. For example, I want to go into inspections in the Coast Guard, specifically the inspections of foreign flagged ships entering US ports. With my education in Naval Architecture, without further higher education my skills are more readily useful in the Coast Guard than anything the Navy could offer, such as EDO which requires at minimum a Masters.


\section{} 
This letter is in support of MIDN 1/C E Hiler's request to cross-commission into the United States Coast Guard. I have known Mr.~Hiler for a year as a faculty advisor to the yard's Satanic Temple religious group, of which he is a founding member. In that context, I am impressed by his deeply questioning, challenging attitude in which he is continually seeking what is ethical and moral and how to pursue nobility in thought and action. I am also familiar with Mr.~Hiler's prior enlisted service, having myself served for five years as a Naval Reactors engineer assigned to Naval Sea Systems Command 08 (Naval Reactors) where I worked on reactor plant fluid systems and new concept submarine and plant designs. 

\section{}
I have not had the pleasure of teaching Mr.~Hiler in one of my classes, but he is doing well as a Naval Architecture and Marine Engineering major and is ahead of his matrix. I have observed all of our students who also completed Nuclear Power School and prototype training do well in the classroom and make good use of their applied, hands-on training with an operating nuclear reactor. On every administrative task I have seen him engaged in, including ECA paperwork and items related to this cross-commissioning paperwork, he has taken the initiative. I believe all of these bode well for his future as an officer wherever he serves.

\section{}
I understand Mr.~Hiler wishes to use his Naval Architecture and Marine Engineering major and hands on engineering experience in service of the United States Coast Guard, in the vital role of inspection and survey, especially of foreign flag ships, to ensure force protection, homeland security, and compliance with customs, bio-security, marine safety, and environmental regulations. A comparable Navy role might be as an Engineering Duty Officer, but after long schooling and prior service already, Mr.~Hiler does not seek a master's degree. I try to encourage nuclear accessions at USNA and often do practice nuclear power interviews for midshipmen, but I find many who were prior-enlisted nuclear field petty officers ultimately choose other commissioning paths; these highly trained individuals are of most benefit to the federal service in roles that fit their long-term goals. 

\section{}
Do you want me to say anything about religious tolerance/climate? 

%\respectfully{}
\noclosing{}
\signature{\includesignature}
\signature{D EVANGELISTA}

\end{document}